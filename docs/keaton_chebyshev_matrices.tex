\documentclass{article}
\usepackage[paperheight=11in, paperwidth=8.5in, margin=2in]{geometry}
\usepackage{amsmath, amssymb}

\begin{document}

\section{Chebyshev matrices}

We begin with the trigonometric definitions of the Chebyshev polynomials, using the change of variables $x = \cos(\theta)$:

\begin{itemize}
    \item 1-st kind:
    
    \begin{equation}
    T_n = \cos(n \theta)
    \end{equation}
    
    \item 2-nd kind:
    
    \begin{equation}
    U_n = \frac{\sin((n+1) \theta)}{\sin(\theta)}
    \end{equation}
\end{itemize}

We observe that the first derivatives of $\{T_n\}$ can be sparsely expressed in terms of $\{U_n\}$ (derived in \S \ref{deriv_diff}):

\begin{equation}
\partial_x T_n = n U_{n-1}
\end{equation}

We can compute the type-1 Chebyshev coefficients of a function efficiently via the FFT, but we wish to solve the first-order Tau system in terms of the type-2 coefficients due to the sparsity of this derivative expression.  To do so we need to construct the following matrices:

\subsection{The evaluation matrix}

First we define $E$, the $T$-to-$U$ evaluation matrix:

\begin{itemize}
    \item \textbf{Inputs}: type-1 Chebyshev coefficients of a function $f$
    \item \textbf{Outputs}: type-2 Chebyshev coefficients of $f$
\end{itemize}

This matrix encodes the linear transformation (derived in \S \ref{deriv_eval}):

\begin{equation}
T_n = \frac{U_n - U_{n-2}}{2}
\end{equation} 

The matrix looks like:

\begin{equation*}
\renewcommand*{\arraystretch}{1.5}
E =
\begin{pmatrix}
1 & \cdot & -\frac{1}{2} & \cdot & \cdot & \cdot \\
\cdot & \frac{1}{2} & \cdot & -\frac{1}{2} & \cdot & \cdot \\
\cdot & \cdot & \frac{1}{2} & \cdot & -\frac{1}{2} & \cdot \\
\cdot & \cdot & \cdot & \frac{1}{2} & \cdot & \ddots \\
\cdot & \cdot & \cdot & \cdot & \frac{1}{2} & \cdot \\
\cdot & \cdot & \cdot & \cdot & \cdot & \ddots
\end{pmatrix}
\end{equation*}

Note: $U_{-n} = - U_{n-2}$ and $U_{-1} = 0$ by the trigonometric definitions. 

\subsection{The differentiation matrix}

Next we define $D$, the $T$-to-$U$ differentiation matrix:

\begin{itemize}
    \item \textbf{Inputs}: type-1 Chebyshev coefficients of a function $f$
    \item \textbf{Outputs}: type-2 Chebyshev coefficients of the derivative $\partial_x f$
\end{itemize}

This matrix encodes the linear operation (derived in \S \ref{deriv_diff}):

\begin{equation}
\partial_x T_n = n U_{n-1}
\end{equation}

The matrix looks like:

\begin{equation*}
\renewcommand*{\arraystretch}{1.2}
D =
\begin{pmatrix}
\cdot & 1 & \cdot & \cdot & \cdot & \cdot \\
\cdot & \cdot & 2 & \cdot & \cdot & \cdot \\
\cdot & \cdot & \cdot & 3 & \cdot & \cdot \\
\cdot & \cdot & \cdot & \cdot & 4 & \cdot \\
\cdot & \cdot & \cdot & \cdot & \cdot & \ddots \\
\cdot & \cdot & \cdot & \cdot & \cdot & \cdot
\end{pmatrix}
\end{equation*}

\subsection{The first multiplication matrices}

Next we define $N_p$, the $T$-to-$T$ multiplication matrices:

\begin{itemize}
    \item \textbf{Inputs}: type-1 Chebyshev coefficients of a function $f$
    \item \textbf{Outputs}: type-1 Chebyshev coefficients of the product $(T_p \cdot f)$
\end{itemize}

These matrices encode the linear operations (derived in \S \ref{deriv_mult1}):

\begin{equation}
T_p \cdot T_n = \frac{T_{n+p} + T_{n-p}}{2}
\end{equation}

The first few matrices look like:

\begin{equation*}
N_0 = I
\end{equation*}

\begin{equation*}
\renewcommand*{\arraystretch}{1.2}
N_1 =
\begin{pmatrix}
\cdot & \frac{1}{2} & \cdot & \cdot & \cdot & \cdot \\
1 & \cdot & \frac{1}{2} & \cdot & \cdot & \cdot \\
\cdot & \frac{1}{2} & \cdot & \frac{1}{2} & \cdot & \cdot \\
\cdot & \cdot & \frac{1}{2} & \cdot & \frac{1}{2} & \cdot \\
\cdot & \cdot & \cdot & \frac{1}{2} & \cdot & \ddots \\
\cdot & \cdot & \cdot & \cdot & \ddots & \cdot
\end{pmatrix}
\end{equation*}

\begin{equation*}
\renewcommand*{\arraystretch}{1.2}
N_2 =
\begin{pmatrix}
\cdot & \cdot & \frac{1}{2} & \cdot & \cdot & \cdot \\
\cdot & \frac{1}{2} & \cdot & \frac{1}{2} & \cdot & \cdot \\
1 & \cdot & \cdot & \cdot & \frac{1}{2} & \cdot \\
\cdot & \frac{1}{2} & \cdot & \cdot & \cdot & \ddots \\
\cdot & \cdot & \frac{1}{2} & \cdot & \cdot & \cdot \\
\cdot & \cdot & \cdot & \ddots & \cdot & \cdot
\end{pmatrix}
\end{equation*}

\begin{equation*}
\renewcommand*{\arraystretch}{1.2}
N_3 =
\begin{pmatrix}
\cdot & \cdot & \cdot & \frac{1}{2} & \cdot & \cdot \\
\cdot & \cdot & \frac{1}{2} & \cdot & \frac{1}{2} & \cdot \\
\cdot & \frac{1}{2} & \cdot & \cdot & \cdot & \ddots \\
1 & \cdot & \cdot & \cdot & \cdot & \cdot \\
\cdot & \frac{1}{2} & \cdot & \cdot & \cdot & \cdot \\
\cdot & \cdot & \ddots & \cdot & \cdot & \cdot
\end{pmatrix}
\end{equation*}

Note: $T_{-n} = T_n$ by the trigonometric definition.

\subsection{The second multiplication matrices}

Next we define $M_p$, the $U$-to-$U$ multiplication matrices:

\begin{itemize}
    \item \textbf{Inputs}: type-2 Chebyshev coefficients of a function $f$
    \item \textbf{Outputs}: type-2 Chebyshev coefficients of the product $(T_p \cdot f)$
\end{itemize}

These matrices encode the linear operations (derived in \S \ref{deriv_mult2}):

\begin{equation}
T_p \cdot U_n = \frac{U_{n+p} + U_{n-p}}{2}
\end{equation}

The first few matrices look like:

\begin{equation*}
M_0 = I
\end{equation*}

\begin{equation*}
\renewcommand*{\arraystretch}{1.2}
M_1 =
\begin{pmatrix}
\cdot & \frac{1}{2} & \cdot & \cdot & \cdot & \cdot \\
\frac{1}{2} & \cdot & \frac{1}{2} & \cdot & \cdot & \cdot \\
\cdot & \frac{1}{2} & \cdot & \frac{1}{2} & \cdot & \cdot \\
\cdot & \cdot & \frac{1}{2} & \cdot & \frac{1}{2} & \cdot \\
\cdot & \cdot & \cdot & \frac{1}{2} & \cdot & \ddots \\
\cdot & \cdot & \cdot & \cdot & \ddots & \cdot
\end{pmatrix}
\end{equation*}

\begin{equation*}
\renewcommand*{\arraystretch}{1.2}
M_2 =
\begin{pmatrix}
-\frac{1}{2} & \cdot & \frac{1}{2} & \cdot & \cdot & \cdot \\
\cdot & \cdot & \cdot & \frac{1}{2} & \cdot & \cdot \\
\frac{1}{2} & \cdot & \cdot & \cdot & \frac{1}{2} & \cdot \\
\cdot & \frac{1}{2} & \cdot & \cdot & \cdot & \ddots \\
\cdot & \cdot & \frac{1}{2} & \cdot & \cdot & \cdot \\
\cdot & \cdot & \cdot & \ddots & \cdot & \cdot
\end{pmatrix}
\end{equation*}

\begin{equation*}
\renewcommand*{\arraystretch}{1.2}
M_3 =
\begin{pmatrix}
\cdot & -\frac{1}{2} & \cdot & \frac{1}{2} & \cdot & \cdot \\
-\frac{1}{2} & \cdot & \cdot & \cdot & \frac{1}{2} & \cdot \\
\cdot & \cdot & \cdot & \cdot & \cdot & \ddots \\
\frac{1}{2} & \cdot & \cdot & \cdot & \cdot & \cdot \\
\cdot & \frac{1}{2} & \cdot & \cdot & \cdot & \cdot \\
\cdot & \cdot & \ddots & \cdot & \cdot & \cdot
\end{pmatrix}
\end{equation*}

Note: Even though the expressions look similar, $M_p \neq N_p$ because $U_{-n} = -U_{n-2}$ while $T_{-n} = T_n$.

\section{Chebyshev operations}

\subsection{Individual operations}

Given $X$, the type-1 Chebyshev coefficients of a function $f$, we can compute:

\begin{itemize}
    \item The type-2 Chebyshev coefficients of $f$:
    
    \begin{equation*}
    E.X
    \end{equation*}
    
    \item The type-2 Chebyshev coefficients of $\partial_x f$:
    
    \begin{equation*}
    D.X
    \end{equation*}
    
    \item The type-1 Chebyshev coefficients of $(T_p \cdot f)$:
    
    \begin{equation*}
    N_p.X
    \end{equation*}
    
    \item The type-2 Chebyshev coefficients of $(T_p \cdot f)$:

    \begin{equation*}
    M_p.E.X
    \end{equation*}
    
    \begin{equation*}
    E.N_p.X
    \end{equation*}

    Since truncation and multiplication do not commute, however, these methods do not produce identical results.  The former will yield the truncated type-2 expansion of the product, while the latter will yield the type-2 coefficients corresponding to the truncated type-1 expansion of the product.  

    \item The type-2 Chebyshev coefficients of $(T_p \cdot \partial_x f)$:
    
    \begin{equation*}
    M_p.D.X
    \end{equation*}
\end{itemize}

\subsection{Combining the operations}

Given $X$, the type-1 Chebyshev coefficients of a function $f$, we can compute the product

\begin{equation*}
C(x) \cdot f
\end{equation*}

\noindent using combinations of these matrix operations.  First, we expand the coefficient $C(x)$ to $P$-th order as a type-1 Chebyshev series:

\begin{equation*}
C(x) = \sum_{p=0}^P c_p T_p(x)
\end{equation*}

The properly truncated type-2 Chebyshev coefficients of the product are simply given by

\begin{equation}
\sum_{p=0}^P c_p M_p.E.X
\end{equation}


\section{Matrix derivations}

\subsection{The evaluation expression}\label{deriv_eval}

\begin{align*}
U_n &= \frac{\sin((n+1) \theta)}{\sin(\theta)} \\
&= \frac{\sin(\theta)}{\sin(\theta)} \cos(n \theta) + \cos(\theta) \frac{\sin(n \theta)}{\sin(\theta)} \\
&= T_n + T_1 \cdot U_{n-1}
\end{align*}

\begin{align*}
U_{n-2} &= \frac{\sin((n-1) \theta)}{\sin(\theta)} \\
&= \frac{\sin(-\theta)}{\sin(\theta)} \cos(n \theta) + \cos(-\theta) \frac{\sin(n \theta)}{\sin(\theta)} \\
&= -T_n + T_1 \cdot U_{n-1}
\end{align*}

\begin{equation*}
\therefore U_n - U_{n-2} = 2 T_n \quad \quad \square
\end{equation*}
    
\subsection{The differentiation expression}\label{deriv_diff}

\begin{align*}
\partial_x T_n &= -n \sin(n \theta) \left(\frac{-1}{\sin(\theta)}\right) \\
&= \frac{n \sin(n \theta)}{\sin(\theta)} \\
&= n U_{n-1} \quad \quad \square
\end{align*}

\subsection{The first multiplication expression}\label{deriv_mult1}

\begin{align*}
T_{n+p} &= \cos((n+p) \theta) \\
&= \cos(n \theta) \cos(p \theta) - \sin(n \theta) \sin(p \theta) \\
&= T_n \cdot T_p - \sin(n \theta) \sin(p \theta)
\end{align*}

\begin{align*}
T_{n-p} &= \cos((n-p) \theta) \\
&= \cos(n \theta) \cos(p \theta) + \sin(n \theta) \sin(p \theta) \\
&= T_n \cdot T_p + \sin(n \theta) \sin(p \theta)
\end{align*}

\begin{equation*}
\therefore T_{n+p} + T_{n-p} = 2 T_n \cdot T_p \quad \quad \square
\end{equation*}

\subsection{The second multiplication expression}\label{deriv_mult2}

\begin{align*}
U_{n+p} &= \frac{\sin((n+p+1) \theta)}{\sin(\theta)} \\ 
&= \frac{\sin((n+1) \theta)}{\sin(\theta)} \cos(p \theta) + \cos((n+1) \theta) \frac{\sin(p \theta)}{\sin(\theta)} \\ 
&= U_n \cdot T_p + T_{n+1} \cdot U_{p-1}
\end{align*}

\begin{align*}
U_{n-p} &= \frac{\sin((n-p+1) \theta)}{\sin(\theta)} \\ 
&= \frac{\sin((n+1) \theta)}{\sin(\theta)} \cos(-p \theta) + \cos((n+1) \theta) \frac{\sin(-p \theta)}{\sin(\theta)} \\ 
&= \frac{\sin((n+1) \theta)}{\sin(\theta)} \cos(p \theta) - \cos((n+1) \theta) \frac{\sin(p \theta)}{\sin(\theta)} \\ 
&= U_n \cdot T_p - T_{n+1} \cdot U_{p-1}
\end{align*}

\begin{equation*}
\therefore U_{n+p} + U_{n-p} = 2 U_n \cdot T_p \quad \quad \square
\end{equation*}

\end{document}

\begin{equation*}
\renewcommand*{\arraystretch}{1.2}
0 =
\begin{pmatrix}
\cdot & \cdot & \cdot & \cdot & \cdot & \cdot \\
\cdot & \cdot & \cdot & \cdot & \cdot & \cdot \\
\cdot & \cdot & \cdot & \cdot & \cdot & \cdot \\
\cdot & \cdot & \cdot & \cdot & \cdot & \cdot \\
\cdot & \cdot & \cdot & \cdot & \cdot & \cdot \\
\cdot & \cdot & \cdot & \cdot & \cdot & \cdot
\end{pmatrix}
\end{equation*}
